\chapter{Filtrační bubliny v kontextu dalších vědeckých teorií}
\label{chapter:filtracni-bubliny-v-kontextu}
    Filtrační bubliny nejsou osamocenou teorií. Na jejich existenci je navázá\-no hned několik dalších fenoménů, které různí autoři zmiňují v přímé souvislosti. Některé z nich mají přímý dopad na vznik filtračních bublin a jiné jsou spíše důsledkem jejich vzniku. Patří mezi ně například (skupinová) polarizace, homofilie, konfirmační zkreslení nebo teorie selektivní expozice.~\citep{Sunstein07, Bruns, Pariser2011, messing, Foth} 
    
    Tyto teorie mají mnoho společného a na první pohled by se mohlo dokonce zdát, že jsou totožné. Jejich definování a pochopení je důležité nejen kvůli jejich úzkému propojení a frekvenci, ve které se objevují v souvislosti s filtračními bublinami, ale také vlivu, který mohou mít na rozhodování a konsensus v otázkách klimatické změny.
    
    Přestože kolem klimatické krize panuje na racionální/vědecké úrovni téměř stoprocentní konsensus, je v konečném důsledku na společnosti, aby jednala. Toto téma je ale ve veřejném měřítku nemožné diskutovat do detailu a racionální diskuze na straně jedné je znemožněna existencí výše zmíněných fenoménů na straně druhé. Například pokud klimatičtí skeptici komunikují v uzavřených skupinách, které jsou založeny na strukturách konfirmačního zkreslení, a klimatičtí vědci vedou pouze racionální diskuzi, je potřeba mnohem většího racionálního informačního tlaku, aby se celkový názor společnosti změnil ve prospěch klimatické krize. Tento princip lze ilustrovat na Brexitu, kde podporovatelé byli silně posílení konfirmačním biasem, kdežto u odpůrců byl tento model zpozorován jen málo. Za předpokladu, že jsou tyto výše popsané mechanismy dobře pochopeny, hrozí jejich potenciální využití a zneužití pro druh sociálního inženýrství - třeba stranami lobujícími proti klimatické krizi.~\citep{muller2020filter}


\section{Polarizace názorů a homofilie}
\label{sec:polarizace-homifilie}
    Vznik internetu nám poskytnul zcela nové možnosti v navazování sociálních kontaktů. Jako globální nástroj nám otevřel možnost komunikace s lidmi širšího názorového spektra, než bylo dosud možné a propojil obyvatele z rozmanitého sociálního prostředí po celém světě. Z toho by se dalo usuzovat, že tato nová setkání umožní rozšíření obzorů jednotlivce a přispějí k názorové diversitě (a tedy zabrání vzniku filtračních bublin).~\citep{Davies} 
    
    Tato teorie však naráží na takzvanou homofilii. Tendenci tíhnout k lidem, kteří jsou nám podobní - mají podobné názory. Homofilii ovlivňují nejen rodinné vazby, ale také například geografie, instituce, které navštěvujeme (škola, práce, zájmové kroužky...), kognitivní procesy atd.~\citep{McPherson} 
    
    \setlength\parskip{5mm}
    
    Podle~\cite{McPherson} je homofilie \textit{„... princip, kdy se objevuje kontakt mezi podobnými lidmi s větší frekvencí než mezi lidmi odlišnými. Všudypřítomná skutečnost homofilie znamená, že kulturní, behaviorální, genetické nebo materiální informace, které proudí mezi sítěmi, mají tendenci být lokálního charakteru. Homofilie naznačuje, že vzdálenost ve smyslu sociálních charakteristik se odráží ve vzdálenosti sítě, množství uzlů, přes které musí informace cestovat, aby se dostala k jednotlivci. Také navrhuje, že jakákoliv sociální entita, která je do velké míry závislá na sítích pro jejich schopnost přenosu, se bude nacházet v sociálním prostoru a bude dodržovat jistá základní pravidla dynamiky během interakce s dalšími entitami v ekologii sociálních norem. “} \footnote{Přeloženo z:  Homophily is the principle that a contact between similar people occurs at a higher rate than among dissimilar people. The pervasive fact of homophily means that cultural, behavioral, genetic, or material information that flows through net-works will tend to be localized. Homophily implies that distance in terms of social characteristics translates into network distance, the number of relationships through which a piece of information must travel to connect two individuals. It also implies that any social entity that depends to a substantial degree on networks for its transmission will tend to be localized in social space and will obey certain fundamental dynamics as it interacts with other social entities in an ecology of social forms.}
    
    Homofilie jako axiom, který říká, že podobnost vytváří nové vztahy, je východ-iskem pro dnešní vědu o sítích. Tato „láska k podobnému“ a dobře známému tak paradoxně uzavírá svět internetu, který měl být ve své podstatě otevřeným prostorem pro komunikaci. Kyberprostor se tak stává jen pouhou sérií echo chambers.~\citep{apprich2018}
    \setlength\parskip{0mm}
    
    Homofilie je v souvislosti se sociálními sítěmi spojována s polarizací. Oba tyto fenomény jsou považovány za úzce související. Jejich vztah však funguje pouze jednostranně – homofilie způsobuje polarizaci a na druhou stranu polarizace nevede k homofilii. Tento vztah však nemusí být vždy zcela daný. Například v případě zkoumání vlivu „tweetů“ o klimatické krizi se ukázalo, že pokud daná informace nemá dostatečnou kredibilitu, homofilie má naopak negativní vliv na polarizaci názorů. V tomto případě byly proto Tweety, které vyjadřovaly anti\-klimatický charakter, shledány jako nevěrohodné~\cite{Samantray}. Zjednodušeně by se tedy dalo říci, že homofilie je spíše příčinou polarizace (a vzniku filtračních bublin) a polarizace naopak důsledkem. 
    
    \setlength\parskip{5mm}
    
    \textit{„K polarizaci typicky dochází, když jsou lidé rozděleni do dvou táborů s protichůdnými názory – například lidé, kteří jsou „pro-volbu“ (např. věří v právo podstoupit potrat) a „pro-život“ (např. věří, že potraty by měly být nelegální).“} \footnote{Přeloženo z: It typically happens when people become divided into groups with opposing perspectives –
    for example, people who are “pro-choice” (i.e., who believe in a right to have an abortion) and “pro-life” (i.e.,
    who believe abortion should be illegal).}~\citep{Nelimarkka} 
    
    Pokud bychom obě názorové skupiny umístili na likertovu škálu a polovina dotázaných by se nacházela na straně „pro-volbu“  (číslo „1“ na škále) a druhá polovina lidí by se postavila na stranu „pro-život“ (číslo „6“ na škále), byly by tyto dvě názorové skupiny považovány za vysoce polarizované. Pokud by se jejich vzájemná vzdálenost na stupnici snížila například na „3“ a „4“ nebo byly jejich hlasy rovnoměrněji rozdistribuovány po celé škále, byla by společnost podle tohoto měřítka výrazně méně polarizovaná.~\citep{Davies}
    
    Podle~\cite{DiMaggio} je polarizace \textit{„...zároveň stav i proces. Polarizace jako stav odkazuje na rozsah vzdálenosti názorů na stejný problém ve vztahu k teoretickému maximu. Polarizace jako proces poukazuje a nárůst názorového rozporu v čase. (s. 693)“}\footnote{Přeloženo z: Polarization is both a state and a process. Polarization as a state refers to the extent to which opinions on an issue are opposed in relation to some theoretical maximum. Polarization as a process refers to the increase in such opposition over time.} 
    
    Polarizace nemusí být definována pouze těmito dvěma pohledy. Dá se na ni dívat z mnoha různých úhlů. Existuje dokonce až devět různých přístupů k definování polarizace.~\citep{Bramson16} 
    
    \setlength\parskip{0mm}
    
    Na sociálních sítích je polarizace způsobená vznikem filtračních bublin a echo chambers a představuje velké riziko pro správně fungující veřejnou debatu.~\citep{Foth} Veřejnost je vystavena informacím, které mají nízkou kvalitu, což můžeme mít negativní dopad. Uživatelé jsou připraveni o možnost poznat různé perspektivy na danou problematiku a na jejich základě tvořit vlastní informovaná rozhodnutí - což je základem deliberativní demokracie. Přísun různorodých názorů (i těch opozitních) je důležitý pro získání konsensu.~\citep{bozdag}
    
    \setlength\parskip{0mm}

\subsection{Skupinová polarizace}
\label{sec:skupinova-polarizace}
    Skupinová polarizace není neobvyklá~\citep{sunstein1999law}. Ovlivňuje naše každo-denní rozhodnutí. Od těch zcela nenápadných, až po ty větší.~\citep{sunstein2009going} Dochází k ní v případě, kdy je stávající názor jednotlivce ještě více posílen rozhodnutím skupiny, které je součástí.~\citep{Isenberg1986GroupPA} To často vede k mnohem většímu posunu směrem k extrému, než by bylo u samotného jedince, který stojí mimo skupinu, běžné. Lidé zastávající už tak extrémní názory se mohou stát ještě extremnějšími. Ostatně predispozici k té největší skupinové polarizaci mají předpoklady lidé, kteří se už na počátku kloní k extrémnějším názorům než většina.~\citep{sunstein1999law}  
    
    Problém skupinové polarizace je obzvláště viditelný na internetu a je zesílený vznikem filtračních bublin (a echo chambers), které umožňují cirkulaci stejného smýšlení uvnitř kyber prostoru a kontakt s jinak izolovanými jedinci. Pokud bude uživatel komunikovat pouze se stejně smýšlejícími lidmi, vyhledávat pouze stránky odpovídající jeho přesvědčení a navrch toho mu bude algoritmem doporučován pouze obsah vycházející z tohoto chování, bude u něj docházet stále k větší extremizaci. Polarizace je proto potenciálně považována za hrozbu demokracie a míru.~\citep{Sunstein07}

%%%%% =====================================

\section{Konfirmační zkreslení}
\label{sec:konfirmacni-zkresleni}
   Na rozdíl od počítačů není lidské usuzovaní nestranné a má tendenci přiklánět se k takovým tvrzením, které mu subjektivně připadají správné a pravdivé. Tato pro lidi charakteristická vlastnost bývá nazývána jako konfirmační zkreslení - nevědomé ohýbání informací takovým způsobem, který potvrdí již stávající pře-svědčení. Právě slovo „nevědomé“ je pro tento termín zásadní, neboť konfirmační zkreslení staví právě na tom, že člověk tak činní nevědomky, bez zřetelného úmyslu, a je pro nějak dokonce skoro nemožné zaujmout nestranné stanovisko, které by mu umožnilo férově zvážit veškeré aspekty dané situace či argumentu. Přirozený postup při ověřování informací je v přímém rozporu s tím, který preferuje věda. Lidé jsou náchylní vyhledávat zdroje, které jejich názor v první řadě potvrdí, nikoliv vyvrátí.~\citep{Nickerson1998ConfirmationBA} 
    
    Konfirmační zkreslení je demonstrováno ve všech fázích vyhledávání informací - rešerše, interpretace i jejich zapamatování. Každý z těchto kroků brání vyvrácení dané hypotézy. Jak už bylo popsáno výše, nejedná se o záměrné zkreslení dat ani o snahu je zdiskreditovat, ale jde o způsob zpracovávání informací, který se děje nevědomky.~\citep{pohl2004cognitive}
    
    Podle~\cite{Jonas2001ConfirmationBI} ke konfirmační zkreslení dochází nejen v případě, kdy jsou účastníkům všechny výsledky vyhledávání předloženy zároveň (má možnost je mezi sebou vzájemně porovnat), ale také pokud jedinec hledá informace sekvenčně. Během čtyř různých experimentů se ukázalo, že konfirmační zkreslení je umocněné, jsou-li informace vyhledávány sekvenčně (účastníci výzkumu při tomto způsobu práce s daty výrazně upřednostňovali články, které byly v souladu s jejich počátečním přesvědčením oproti zdrojům, které s nimi byly v kontradikci). 
    
    Konfirmační zkreslení je možné pozorovat také na sociálních sítích, které jsou také spojovány s filtračními bublinami.~\citep{kowald2018studying}. 
    
    \setlength\parskip{5mm}
    
    \textit{„Filtrační bubliny mají tendenci dramaticky umocňovat konfirmační zkreslení - svým způsobem jsou pro to stvořeny. Přijímání informací, které odpovídají našim představám o světě, je jednoduché a příjemné; příjímání informací, které nás vyzývají abychom přemýšleli zcela novými způsoby nebo zpochybňovali naše stávající předpoklady, je frustrující a těžké. Proto také podporovatelé jedné politické strany mají tendenci vyhýbat se příjmu informací z médií, které jsou určené straně opoziční. Výsledkem je, že informační prostředí, které je založené na signá\-lech „kliknutí“ bude upřednostňovat obsah, který podporuje již existující představy o světe před obsahem, jenž jej zpochybňuje.“} \footnote{Přeloženo z: The filter bubble tends to dramatically amplify confirmation bias—in a way, it’s designed to. Consuming information that conforms to our ideas of the world is easy and pleasurable; consuming information that challenges us to think in new ways or question our assumptions is frustrating and difficult. This is why partisans of one political stripe tend not to consume the media of another. As a result, an information environment built on click signals will favor content that supports our existing notions about the world over content that challenges them.}~\citep{Pariser2011} 
    
    Mechanismus konfirmačního zkreslení je silný a pokud se jedinec dostane do skupiny (ať už Facebookové či jiné), která je postavena na základech tohoto názorového posílení, má také tendenci v této skupině zůstat.~\citep{muller2020filter}
    
    \setlength\parskip{0mm}
%%%%% =====================================
\section{Teorie selektivní expozice}
\label{sec:selektivni-expozice}
    Každá živý organismus na planetě je každou sekundu vystavován enormnímu množství stimulů. Ať už se jedná o vůně, světlo, částice ve vzduchu nebo okolní zvuky. Vnímat všechny tyto stimuly je téměř nemožné. Nejenže k tomu není žádný organismus přizpůsobený, ale nebylo by to ani praktické pro přežití. To platí samozřejmě i pro Homo sapiens. Stejně jako ostatní druhy na to nejsme stavěni. Je tedy naprosto přirozené, že lidé selektivně filtrují, čemu budou věnovat pozornost - ať už se jedná o fyzické vjemy nebo o informace. Některé tyto pomyslné „filtry“ jsou automatické a jiné neúmyslné a nevědomé.~\citep{zillmann2013selective}
    
    Jestliže má člověk v podstatě svého bytí zakódovanou pouze omezenou kapacitu určenou pro zpracovávání informací, orientace v současném toku informací, je pro něj zcela nemožnou a nedosažitelnou metou. Informační přehlcenost\footnote{Informační přehlcenost je situace, kdy existence příliš velkého objemu informací zabraňuje člověku najít požadovanou informaci a to způsobuje problémy při rozhodování.~\citep{Renjith}} se totiž stala symbolem moderní společnosti. Ve věku, který je charakterizován neustále rostoucí explozí informací, se tento problém přehlcenosti stává skutečně hmatatelným. Každou sekundu se internet plní tisícemi informací, které jsou mimo kognitivní schopnosti jakékoliv živé bytosti. Se vznikem sociálních sítí, které duplikují data a dávají prostor pro šíření falešných zpráv, se situace ještě zhoršila.~\citep{Renjith}
    
    Jeden z nejjednodušších způsobů pro jedince, jak se v množství informací zorientovat, je vyhnout se informacím, se kterými zkrátka nesouhlasí. Obecně vzato je vůle dobrovolně se vystavovat určitým informacím považována za vysoce selektivní (princip přijímaný v sociální psychologii, masové komunikaci i socilogii~\citep{FreedmanSears}). To znamená, že stejně jako u výše popsaných pojmů platí u selektivní expozice, že lidé vyhledávají informace, které potvrzují jejich existující přesvědčení a mají tendenci se vyhýbat protichůdným tvrzením.~\citep{Berkowitz}
    
    Toto tvrzení je prakticky definující pro teorii selektivní expozice o které~\cite {FreedmanSears} říkají, že \textit{„Nejsilnějším formou tvrzení selektivní expozice je, že lidé preferují být vystaveni komunikaci, která je v souhlasu s jejich již dříve existujícími názory. Z toho důvodu lidé aktivně uvažují o vyhledávání materiálů, které podporují jejich názory a aktivně se vyhýbají názorům, které je zpochybňují.“}\footnote{Přeloženo z: The strongest form of the selective exposure proposition is that people prefer exposure to communications that agree with their pre-existing opinions. Hence, people are thought actively to seek out material that supports their opinions, and actively to avoid material that challenges them.} (\textcolor{red}{s.197}) 
    
    Selektivní expozice může vznikat za různých okolností. Za podmínky přímého přístupu k sdělovacím prostředkům/médiím vzniká pravděpodobně záměrně nebo „samo-zvolenou“ personalizací.~\citep{cardenal} Na druhou stranu selektivní expozice může být také důsledkem personalizace a filtrování obsahu, a způsobovat tak vznik filtračních bublin~\citep{Pariser2011}. Vzájemná propojenost mezi pojmy zmíněnými v této kapitole je demonstrovaná i zde, neboť jedním z důsledků selektivní expozice je polarizace.~\citep{ZuiderveenBorgesius2016Should}