\setcounter{page}{1}

\addcontentsline{toc}{chapter}{Úvod}
\chapter*{Úvod}
\label{chapter:uvod}
    Španělský sociolog Manuel Castells je spojován s pojmem informační spole\-čnost, kterou později ve své knize The Network Society rozvinul také Nizozemský sociolog Jan van Dijk ~\citep{castells2011rise, van2020network}. Tímto konceptem popisuje Dijk společnost, kde je veškerá výměna informací závislá na sociálních a mediálních sítích, které jsou zároveň pojítkem mezi jednotlivými články komunikace - jednotlivci organizacemi, skupinami. 
    
    Pro informační společnost je charakteristická vysoká intenzita informačního toku proudícího přes média, která udávají tón kultuře ~\citep{van2020network}. Proto se můžeme ptát, nakolik je právě dominantní postavení mediálních a společenských sítí jedním z důvodů, proč se potýkáme s nárůstem vyfabulovaného obsahu, který je nositelem dezinformací, takzvaných fake news~\citep{lazer2018science}, i napříč tomu, že informační společnost by měla být podle ~\citep{van2020network} založena na vědě, racionalitě a reflexivitě.
    
    V množství informací, které skrze média v současnosti proudí, je těžké utvořit si plně zformovaný a informovaný názor. Toto kvantum informací a dat umožňuje (skoro až vyžaduje) médiím stavět se do pozice jakéhosi filtru, který definuje relevantní obsah a určuje jeho narativ. Nejenže tak média zprostředkovávají určitý obraz světa, ale dokonce jej pomáhají utvářet. ~\citep{vattimo1992transparent}
    
    Pro ilustraci v roce 2018 bylo na české mediální scéně zveřejněno přes 80 000 článků na téma migrace v souvislosti s probíhající uprchlickou krizí. To je asi jeden článek na každé dva uprchlíky, kteří dorazili do Evropy skrze Středozemní moře. Pro porovnání: V tom samém období bylo zveřejněno jen okolo 20 000 článků na téma klimatické krize (včetně článků o tajících ledovcích, emisích CO2, atd.). ~\citep{prokop2019slepe}
    
    Vytvořené sociální a mediální struktury spojují lidstvo do klastrů a vytváří tak pomyslné malé světy. Společenství, kde je každý jednotlivec nebo skupina propojená skrze několik prostředníků. Žijeme tak v propojeném světě, kde jsme si mnohem blíže, než kdy před tím - jsme součástí informační společnosti. ~\citep{van2020network} Přesto jsme si v určitém smyslu - jakkoliv to zní otřepaně - navzájem mnohem vzdálenější než kdy dříve. 
    
    To se demonstruje na fungování sociálních sítích, které se stávají důležitým zdrojem pro konzumaci zpráv ~\citep{olmstead2011navigating}.\footnote{Až 52~\% Američanů čerpá informace o zprávách z Facebooku a 72~\% uživatelů Facebooku využívá tuto platformu jako zdroj zpráv ~\citep{shearer2019americans}.} Se sociálními sítěmi filtrujících obsah podle relevantnosti pro daného čtenáře pak může dojít k omezení různorodosti přijímaného sdělení a vzniku filtračních bublin, ze kterých není snadné se vymanit a které nám znemožňují objektivně nahlížet na otázky vnějšího světa ~\citep{Claypool1999CombiningCA, Pariser2011, Foth}. Problémy jako klimatická změna nám pak mohou připadat jako nedůležité ~\citep{kennedy2019us}.\footnote{Studie ukazuje, že 60~\% amerických občanů vnímá klimatickou změnu jako zásadní hrozbu pro kvalitu života ve Spojených státech.}
    
    Otázkou je, zda je vůbec možné vymanit se ze současného systému doporu\-čování obsahu, aniž bychom museli obětovat svoje profily na sociálních sítích. 
    
    Tato diplomová práce sice nenastiňuje jasné východisko, ale prostřednictvím zkoumání mechanismu doporučování stránek na Facebooku se snaží popsat možný vliv tohoto algoritmu na utváření filtračních bublin. Předmětem uskutečněné analýzy jsou stránky, jejíchž hlavní tématem je klimatická krize, která má tendenci společnost polarizovat a kolem níž panuje řada dezinformací \citep{kolmes2011climate, WILLIAMS2015126}. 
    
    Text tedy nenabízí řešení, ale je spíše nezávislým pozorovatelem, jehož poznatky, doufejme, poslouží jako podnět k dalšímu řešení. 
    
    \begin{flushright}
    \textit{„If we allow global warming to proceed, \\
    and to punish us with all the ferocity we have fed it, \\
    it will be because we have chosen that punishment\\
    - collectively walking down a path of suicide.“}\\ \citep{wallace2019uninhabitable} \\
    \end{flushright}



%----------------------------------------------------------------------------