\chapter{Design výzkumu}
\label{chapter:design-vyzkumu}
    Facebook je prudce se rozvíjející platformou, která neustále mění, aktualizuje a zlepšuje algoritmy, které určují, co se bude na stránce dít a jaké bude mít funkce~\citep{hao_2021}. Přestože zástupci této společnosti přislíbili, že budou usilovat o větší transparentnost, je veřejnosti více méně skryto, jak přesně jednotlivé algoritmy fungují .~\citep{satterfield_2020}
    
    Na základě konkrétního předmětu výzkumu je tak méně či více snadné vyvinout metodiku pro danou studii. V průběhu sběru dat navíc může dojít k novým aktualizacím Facebooku, což může způsobit komplikace či úplnou ztrátu potřebných informací. Například v minulosti se design facebookových stránek několikrát změnil a s ním také možnosti přístupu k doporučovaným stránkám~\citep{facebook_2005, facebook_2020}. Například dříve bylo možné dohledat doporučené stránky ke konkrétní stránce za pomocí url adresy\footnote{https://www.facebook.com/pages/\linebreak?ref=page\_suggestions\_on\_liking\_refresh\&
    from pageid=}. Stačilo pouze za poslední rovnítko připsat ID požadované facebookové stránky. Následně se souhrnně zobrazily ostat\-ní, k ní doporučené, stránky. Takto už to dnes nefunguje a při vyzkoušení tohoto postupu je generován stále stejný seznam stránek bez jasné návaznosti na vstupní stránku.   
    Z toho důvodu byla pro cíle tohoto výzkumu vytvořena vlastní metoda postupu sběru dat a jejich vyhodnocení, která odpovídá současným podmínkám a výzvám Facebooku bez přímé replikace jiných studií.

%%------------------------------------------------------------------------
\section{Kategorizace dat}
\label{sec:kategorizace-dat}
    Před tím, než byl započat sběr dat a uskutečněna analýza facebookových doporučení, byly nejdříve definovány stránky, které jsou předmětem této analýzy. Přestože v cíli výzkumu jsou jednoznačně pojmenovány „stránky, které se zbývají klimatickou krizí“, tato definice je příliš obecná a do určité míry se mění v průběhu výzkumu, neboť na stránky se dá pohlížet z různých perspektiv.
    
    Na začátku proto byla potřeba rozdělit si stránky podle fáze, v které se ve výzkumu vyskytují: 
    
    \begin{enumerate}
      \item \textbf{Stránky vstupní (primární)}
    
        Jedná se pravděpodobně o tu nejdůležitější kategorii stránek, která byla vybírána záměrně a manuálně. Význam těchto stránek spočívá v tom, že jsou stavebním kamenem pro celou analýzu - vychází z nich všechna následují\-cí data.
        
        Vstupní stránky byly vybírány v angličtině, a to z toho důvodu, že tento výběr umožnil pracovat s větším množstvím stránek a tak přirozeně i větším množství dat. Práce s anglickými stránkami také otevírá možnost používat širší škálu nástrojů, které jsou často funkční pouze v angličtině. Jazyk byl v tomto ohledu jediným určujícím faktorem. Dále už nebyly stránky filtrovány podle konkrétní země - mezi primárními stránkami se proto nachází například mezinárodní stránka organizace Greenpeace, stránka vztahující se k obyvatelům Austrálie nebo stránka NASA z USA. Do výběru se mohla dostat i stránka jejíž správce se nachází v kterékoliv zemi na světě - jeho rodným jazykem nemusí být angličtina. Země původu tedy není určujícím kritériem.
        
        Aby se stránka dostala do primárního výběru, musí se nejen věnovat klimatické krizi, potažmo klimatické změně v antropocentrickém smyslu, ale důležitý je také postoj, který k tomuto tématu zaujímá. 
        
        Vstupní stránky, stejně jako sekundární a terciální, se dělí podle svého postoje ke klimatické krizi. Na jedné straně stojí \textbf{\uv{pro}} = s pozitivním postojem ke klimatické krizi - respektive takové, které souhlasí s existencí klimatické krize, která je způsobená lidmi. Na straně druhé jsou stránky s opačným postojem, tedy \textbf{\uv{proti}} = nesouhlasí s existencí klimatické krize, která je způsobená lidmi. Tyto stránky lze často (ne výlučně) považovat za dezinformační.
        
        Primární stránky byly vybírány tak, aby co nejpřesněji reprezentovaly dané postoje. Proto se jedná o stránky (především na straně \uv{proti}), které mají až extrémním vztah ke klimatické krizi - v případě \uv{proti} jsou to často stránky, které jsou až dezinformační a svůj postoj ke klimatické krizi velmi explicitně projevují - například se jej nesnaží sofistikovaně zaobalit do vědeckých dat. Dobrou ukázkou je stránka „Global Warming, Climate Change, whatever it's called is a scam“\footnote{Globální oteplování, klimatická změna nebo jak se to jmenuje je podvod.}, která svůj vztah k tématu klimatické změny vyjadřuje už ve svém názvu. 
        
        Stránky \uv{pro} jsou charakterizovány zejména tím, že se většinou jedná o stránky známých organizací (např. Greenpeace, NASA, Nature). Ale také jsou tyto stránky z větší části nositeli „modré fajfky“, která je oficiálním označením Facebooku pro pravost určité stránky. Ne ve smyslu pravosti informací, ale ve smyslu identity dané osoby či organizace. Stejné označení dostala i stránka CFACT, která se staví proti klimatické krizi.). I mezi vstupními stránkami \uv{pro} klimatickou krizi se nachází stránky vyjadřující svůj postoj. Například: Climate Change Is Real\footnote{Klimatická změna je opravdová.}. 
        
      \item \textbf{Stránky první úrovně (sekundární)} 
      
      Stránky první úrovně jsou takové stránky, které byly uvedeny jako doporučené na vstupních (primárních) stránkách. Mezi těmito stránkami se nacházely i takové, které jsou s klimatickou změnou zcela nesouvisející, proto byly v pozdější fázi výzkumu roztříděny, kategorizovány a případně vyřazeny.
      
      \item \textbf{Stránky druhé úrovně (terciární)} 
      
      Terciární stránky navazují na sekundární stránky. Jsou to tedy stránky, které byly uvedeny na předchozích stránkách první úrovně jako doporučené. Stejně tak byly roztříděny podle jejich souvislosti a vztahu ke klimatické krizi. Spolu se sekundárními stránkami jsou určujícím ukazatelem pro vznik nebo potlačení filtračních bublin. 
      \end{enumerate}
    
      Pokud u některých stránek první nebo druhé úrovně není na první pohled zcela zřetelné, zda patří do skupiny \uv{proti} nebo \uv{pro}, je vždy určující, zda klimatickou změnu v antropocentrickém pojetí přijímají nebo popírají. Například vyjadřování podpory nukleární energii nebo dokonce fosilním palivům ještě nutně neznamená, že je daná stránka proti - pouze to poukazuje na nekonvenční (oproti diskurzu klimatických aktivistů) přístup k řešení \citep{plumer_fountain_albeck-ripka_2018,carrington_2020}.
  
%%------------------------------------------------------------------------
\section{Nástroje analýzy a zdroje dat}
\label{sec:nastroje-analyzy}

Ke stažená dat a zpracování výsledků bude použito několik různých nástrojů. Tím uživatelsky nejjednodušším a nejběžnějším je Excel. Naopak tím uživatelsky nejnáročnějším je programovací jazyk Python\footnote{Stažení dat a částečné zpracování výsledků bylo umožněno spoluprací s programátorem.}. Dalším speciálním nástrojem je CrowdTangle, který posloužil ke stažení dodatečných dat z Facebooku.%i speciálními nástroji, které budou více popsány, jsou CrowdTangle a Voyant Tools. 

%\begin{enumerate}
   % \item Voyant tools
    
    %Voyant tools je volně dostupný software pro čtení a analýzu textů, který je možné používat skrze webovou aplikaci. Jeho kód je takzvaný „open-source“ (otevřený) a dostupný na serveru GitHub\footnote{https://github.com/sgsinclair/Voyant}. Je to původně akademický projekt, který je určený nejen pro akademiky, studenty digital humanities, ale i pro širokou veřejnost. 
    
    %Voyant Tools má téměř třicet různých použitelných nástrojů a umožňuje nejen počítačem asistovanou textovou analýzu, ale jeho funkce mohou být také přidány na různé webové stránky - blogy, časopisy a další. 
    
    %Aplikace pracuje hned s několika „open-source“ knihovnami a zakládá si na dostupnosti a jednoduchosti. Mezi designové principy patří také například modularit, flexibilita a mezinárodnost - Voyant Tools má mezi jinými také českou verzi.~\citep{voyanttools_2021}
    
    \subsection{CrowdTangle}
    
    CrowdTangle původně vznikl v roce 2011 nezávisle na Facebooku a svůj produkt vystavěl na veřejně dostupných API\footnote{Application Programming Interface - softwarový  prostředník, který umožňuje dvou aplikacím mezi sebou komunikovat. \citep{mulesoft_2021}}. Později v roce 2016 se společnost spojila s Facebookem a v současné době je služba považována za jednu z jeho aplikací.~\citep{matt_2016}
    
    CrowdTangle umožňuje jednoduše sledovat, analyzovat a následně pochopit, jak se obsah na sociálních sítích šíří a jaké jsou trendy nejen na Facebooku, ale také na Instagramu nebo Redditu. 
    
    Přesněji tento nástroj sleduje, kdy byl obsah zveřejněn, z jaké stránky či veřejného účtu byl zveřejněn nebo na jako stránku byl uveřejněn. Dále sleduje interakce, shlédnutí a sdílení. 
    
    Veřejně, bez registrace, je kromě rozšíření do prohlížeče, které umožňuje zjistit, kdo sdílel daný článek, dostupný také živý přehled, který umožňuje sledovat vývoj a konverzaci kolem specifických témat na Facebooku v reálném čase.
    
    Od roku 2019 CrowdTangle přizval do aplikace akademiky a vědce, aby jim umožnil lépe pochopit, jak se obsah na sociálních sítích šíří. Od toho si slibuje větší transparentnost Facebooku, která otevře online konverzaci a mimo jiné povede k větší bezpečnosti, spolehlivosti a přesnosti sociálních sítí. ~\citep{bleakley_2021}

%\end{enumerate}

%%------------------------------------------------------------------------
\section{Sledované položky}
  Ve chvíli, kdy byla shromážděna všechna potřebná data, bylo přistoupeno k samotné analýze. V první řadě je sledováno, jaké je zastoupení doporučovaných stránek a jakým způsobem jsou doporučovány. Konkrétně je sledováno:
  
\begin{itemize}
    \item Kolik je celkově zastoupeno stránek \uv{pro} a kolik \uv{proti}.
    \item Jak jsou stránky mezi sebou vzájemně propojeny. Neboli zda různorodým doporučováním dochází k prolamování informačních bublin.
    \item Které stránky jsou častěji doporučovány. Zda \uv{pro}, \uv{proti} nebo ekvivalentně. 
\end{itemize}
  
  Pro lepší porozumění sledovaných stránek a tomu, jak by mohl facebookový algoritmus stránky klasifikovat na základě jejich obsahu a interakcí, jsou sledovány především následující ukazatele: 
  
 \begin{itemize}
    \item Četnost interakcí na stránkách \uv{pro} a \uv{proti} 
    \item Druhy interakcí na stránkách \uv{pro} a \uv{proti}
    \item Druhy příspěvků na stránkách \uv{pro} a \uv{proti} (link, vlastní obsah atd.)
\end{itemize}
  
