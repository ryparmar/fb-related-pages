\chapter{Filtrační bubliny}
\label{chapter:filtracni-bubliny}
    Klimatická změna je naléhavým problémem, který vyžaduje celosvětovou pozornost a kooperaci~\citep{stokes2015global}. Přesto existují skupiny obyvatel, které ji nepovažují za reálnou hrozbu. Tento postoj je obzvláště patrný v USA, kde přibližně jeden z deseti Američanů věří, že klimatická krize není významnějším problémem, a 23 \% ji nepovažuje za reálný problém.~\citep{fagan_huang_2020} K tomu bezesporu přispívá i fakt, že jen málo Američanů skutečně rozumí tomu, co z pohledu vědy klimatická změna opravdu znamená, a to vlivem rozsáhlé dezinformační kampaně.~\citep{kolmes2011climate} Tento negativní pohled na tuto globální krizi, který je plný zavádějících informací může být ještě umocněn filtračními bublinami, které podporují vznik dezinformačních skupin.~\citep{Bruns}
    
    Protože mají sociální sítě potenciál pozitivně ovlivňovat společenskou debatu, někteří odborníci doufali, že se stanou prostorem pro otevřenější a plodnější diskuzi, neboť umožní uzavřenějším jedincům s různými osobnostními rysy svobodně se vyjádřit, a přispět tak k rozšíření veřejného diskurzu a odhalení nových perspektiv každodenní, ale i politické diskuze.~\citep{hampton} 
    Tato vize, která byla projektována do sociálních sítí, je staví do role jakéhosi štítu, který jedince anonymizuje, a chrání ho tak od vlivu jeho blízkých, čímž vytváří prostředek pro řešení takzvané spirály mlčení – jedinec potlačí svůj vlastní názor, jestliže věří, že jeho okolí, přátelé, rodina nebo kolegové, nesdílí stejné přesvědčení.~\citep{Noelle} 
    Přestože tento sociologický pojem od Elizabeth Noelle-Neumann vznikl v období 60. a 70. let, tedy ještě v „před internetové době“~\citep{Petersen}, ukazuje se, že jako efekt sociálního vlivu je přítomný i v online prostředí, kde se také může projevovat spirála mlčení. Uživatelé sociálních sítí jsou totiž ochotnější interagovat s příspěvky, které už dříve okomentovali lidé z jejich vlastní komunity~\citep{Nasim}.
    
    
    Zmíněné chování pak může mít na jednotlivých internetových platformách jednoznačný vliv na personalizaci dat. Za pomocí kolaborativního filtrování, kdy algoritmus predikuje zájmy jednotlivců na základě chování jejich vrstevníků, je uživatelům zobrazován pouze obsah korespondující s názorovým přesvědčením jedinců, kteří se nacházejí v jejich vlastním sociálním okruhu.~\citep{Claypool1999CombiningCA}
    
    Algoritmy určené k doporučování obsahu mají mimo jiné za cíl pomáhat uživatelům orientovat se v prostředí, které je přesycené informacemi a ve kterém by navigace jinak mohla být složitá. Využívají k tomu informace plynoucí z uživatelského profilu, zájmy, zvyky, lokaci, záznam o aktivitě na dané platformě (likes, příspěvky atd). Tím zajišťují, že se k jedinci dostane obsah, který je přizpůsobený jemu na míru. Kód, který je zodpovědný za personalizaci, však není zatížený etickým kodexem, ale sleduje zadané cíle (např. zvýšit návštěvnost stránky - toho dosahuje tím, že nabízí relevantní obsah, který zajímá konkrétního návštěvníka).~\citep{Foth} 
    
    Tyto nástroje, založené na datech o uživatelské aktivitě, můžou v důsledku výrazně omezit rozsah názorového spektra, se kterým přijdou uživatelé do styku. Najednou mají možnost zapojit se do konverzace pouze s lidmi stejného názoru. Vznikají tak homogenní komunity, z nichž je „vykázán“ každý, kdo s danou skupinou nesdílí stejné názorové přesvědčení.~\citep{Parsell}
    
    Toto pomyslné rozdělení jedinců do jednotlivých názorových táborů ve svém díle definoval a pojmenoval~\cite{Pariser2012TheFB} jako filtrační bubliny, které blíže popsal následujícím způsobem:
    
    \setlength\parskip{5mm}
    
    \textit{
    „Základní kód, který leží v srdci moderního internetu je jednoduchý. Nová generace internetových filtrů zakládá svá doporučení na věcech, které podle nich máme rádi nebo aktivitách, které jsme v minulosti skutečně udělali my nebo lidé nám podobní. Z nich se následně snaží vycházet. Tyto systémy založené na předpovědi neustále vytváří a vylepšují teorii o tom kdo jsme, a co uděláme příště. Společně tak pro každého z nás vytváří unikátní vesmír informací, který jsem nazval filtrační bublina a, který od základu mění způsob, jakým vnímáme myšlenky a informace. (s. 10)“}\footnote{Přeloženo z originálu: „The basic code at the heart of the new Internet is pretty simple. The new generation of
    Internet filters looks at the things you seem to like—the actual things you’ve done, or the things people like you like—and
    tries to extrapolate. They are prediction engines, constantly creating and refining a theory of who you are and what you’ll do
    and want next. Together, these engines create a unique universe of information for each of us—what I’ve come to call a filter
    bubble—which fundamentally alters the way we encounter ideas and information.“}
    
    
    \cite{Pariser2011} dodává, že tendence vybírat si obsah, který nás oslovuje a vyhýbat se tomu, co naopak ne, tady byla od nepaměti. Na druhou stranu filtrační bubliny s sebou přináší nové aspekty (nebo také „dynamiky“, jak je sám pojmenovává), se kterými jsme se nemuseli dříve potýkat a které nyní zasahují do naší schopnosti svobodně si vybírat obsah: 
    
    \begin{enumerate}
      \item Ve filtračních bublinách jsme osamoceni.
      
      Přestože je technicky zapotřebí více lidí, kteří mezi sebou sdílejí informace na dané téma, aby se vytvořila filtrační bublina, každý jedinec je sám uzavřen ve své vlastní bublině - unikátně vytvořeném personalizovaném prostředí. 
      
      \item Filtrační bublina je neviditelná.
      
      Jakýkoliv algoritmus, který ve větší či menší míře napomáhá k tvorbě filtrační bubliny, je neprůhledný - nevíme, jak přesně funguje a proč nám doporučuje/zobrazuje zrovna tento obsah. Dokonce ani nevíme, jaké vlastnosti do našeho uživatelského profilu projektuje a zda jsou pravdivé nebo nepravdivé. Například rozhodnutí stát se členem určité církve s sebou většinou nese znalost hodnot, pravidel a systému daného společenství. V kontrastu k tomu jsou filtrační bubliny zcela neprůhledné. Kritéria filtrování obsahu jsou pro nás neznámá.
      
      \item Vstup do filtrační bubliny je nedobrovolný.
      
      Algoritmus, který stojí za personalizovaným obsahem je inherentní pro daný web. To znamená, že většinou nemáme možnost se rozhodnout, zda jej chceme zapnout či nikoliv. Je součástí využívané služby a je často využíván k monetizaci obsahu stránky - tvůrci díky němu generují zisky, které jim umožňují web udržet v chodu. To je mimo jiné jeden z důvodů, proč je čím dál těžší se mu vyhnout. 
      
    \end{enumerate}

    \cite{Bruns} ve své knize Are filter bubbles real? uvádí, že je možné jen zřídka najít jasnou definici (jsou totiž málo explicitní a neustále se proměňují v čase) filtračních bublin. Nabízí tedy krom jiných i svou vlastní definici filtračních bublin:
    
    \uv{\textit{Filtrační bubliny vznikají, když se skupina účastníků, nezávislá na struktuře dané sítě, na které komunikuje s ostatními členy, rozhodne přednostně komunikovat vzájemně, a vyloučí tak z diskuze lidi, kteří se nepohybují v jejich kruhu. Čím konzistentněji se ubírají k takovýmto praktikám, tím pravděpodobnější je, že názory a informace právě těchto účastníků budou spíše cirkulovat pouze mezi jejich členy namísto nových informací, které by mohly proniknout zvenčí, tedy z prostředí mimo jejich uskupení. }}\footnote{A filter bubble emerges when a group of participants, independent of the underlying network
    structures of their connections with others, choose to preferentially communicate with each other, to
    the exclusion of outsiders. The more consistently they adhere to such practices, the more likely it is that
    participants’ own views and information will circulate amongst group members, rather than
    information introduced from the outside.}~\citep{Bruns17}
    
    
    Přestože~\cite{Pariser2011} i~\cite{Bruns} ve svých definicích popisují stejný jev, kde účastníci zůstávají uzavřeni v bublině, ve které jsou obklopeni stejnými názory, jejich vymezení filtračních bublin se liší v tom, kdo stojí v centru tohoto procesu. Pariser staví do popředí algoritmus, který filtruje informace na základě dat o uživateli, zatímco Bruns píše o uživateli, který si vybírá, s kým bude komunikovat. Brunsův pohled tedy více navazuje (na rozdíl od Parisera) na teorii spirály mlčení, ač přítomnost algoritmu a jeho podíl na vzniku filtračních bublin je i v tomto případě nepopiratelná. 
    
    \setlength\parskip{0mm}
    
    Toto dvojí vnímání způsobu, jakým se člověk může dostat do středu filtrační bubliny, by mohlo být kategorizováno pod pojmy „samo-zvolená“ personalizace a „předurčená“ personalizace.\footnote{V originálu jako: self-selected personalization a pre-selected personalization} 
    
    „Samo-zvolená“ personalizace ve své podstatě znamená, že se jedinec dobrovolně vyhýbá obsahu, který je v rozporu s jeho vlastním přesvědčením a věnuje svoji pozornost pouze takovému obsahu, který je s jeho přesvědčením v souladu - stejně jako navrhuje~\cite{Bruns17}. 
    
    Na druhé straně potom stojí „předurčená“ personalizace, která je realizována webovými platformami za účelem filtrování uživatelského obsahu často bez vědomí a souhlasu uživatele. „Předurčená“ personalizace pak můžu být rozdělena na vědomou a nevědomou podle toho, zda si uživatele vědomě danou platformu vybere s cílem získávat tento personalizovaný obsah - tento typ personalizace pak odpovídá spíše definici~\cite{Pariser2011}.~\citep{BrunsSpringer}
    
    Pro účely této práce bude chápána personalizace jako „předurčená“ s odpovídající definicí podle~\cite{Pariser2011}.
    
    K personalizaci je přistupováno s velkou obezřetností neboť není zcela jasné, zda nemůže mít negativní dopad na společnost, demokracii, autonomii jedinců apod. Ze současného veřejného diskurzu nevyplývá jasný empirický důkaz, který by objasnil, zda jsou tyto obavy podložené nebo přehnané. 
~\citep{ZuiderveenBorgesius2016Should} 
    
    \setlength\parskip{0mm}
    
    Podobně existenci filtračních bublin, spolu s jejich potenciálně negativním vlivem, který mohou mít na jedince, netřeba podceňovat. Nevzbuzují však nutně obavy u všech uživatelů sociálních sítí a dokonce ani u všech odborníku na tuto problematiku.~\citep{Grossetti} 
    
    Výzkumy prováděné na téma filtračních bublin jsou často realizovány ve Spojených státech a pracují s tamním systémem dvou politických stran, který nemusí být vždy stoprocentně aplikovatelný na státy se systémem více stran~\citep{ZuiderveenBorgesius2016Should}. Dobrým příkladem jsou níže zmíněné studie~\citep{Dubois, Barbera, NECHUSHTAI2019298}, které pracovaly s respondenty dvou politických názorových skupin - konzervativci a liberálové - a zjistily rozdílné riziko ovlivnění filtračními bublinami pro každou z těchto skupin, které může být ovlivněno hned několika faktory.
    
    Tito autoři nepřímo naznačují, že jsou filtrační bubliny poněkud nafouknuté nad své pravé rozměry a ve skutečnosti nemají takový vliv na veřejnost. Šance, že bude jedinec uvězněn ve filtrační bublině se může odvíjet od zájmů daného jednotlivce a jeho celkové otevřenosti diskutovat o tématech s lidmi opačného názoru. Například lidé, kteří se zajímají o politiku a sledují širokou škálu médií, mají velkou šanci se filtračním bublinám vyhnout.~\citep{Dubois} Stejně jako zkoumáním ideologických preferencí mezi 3.8 miliony Twitterových účtů bylo zjištěno, že jsou liberálové mnohem přístupnější k diskuzi širšího spektra hodnot než konzervativci~\citep{Barbera}.
     
    V kontrastu k těmto studiím~\cite{NECHUSHTAI2019298} dochází k závěru, že liberálům stejně jako konzervativcům služba Google News doporučuje média, která typicky cílí na opačnou cílovou skupinu (např. články konzervativních novin The Washington Times se zobrazovaly liberálním účastníkům) a seznam výsledků zpravodajských webů je pro obě skupiny téměř totožný. Politické preference v tomto případě nemají téměř žádný vliv na výsledné doporučení.
    
    Podle~\cite{ZuiderveenBorgesius2016Should} personalizovaný obsah netvoří pro většinu uživa\-telů hlavní zdroj informací a dokud tato skutečnost přetrvá, není důvod obávat se větších negativních dopadů filtračních bublin na veřejnost. K podobnému závěru dochází také~\cite{Krafft2017}, který tvrdí, že v Německu se díky smíšenému mediálnímu využití výrazně snižuje možnost uzavření jedince v jeho vlastní názorové bublině. 
    
    Snaha zabránit vzniku filtračních bublin by měla vycházet nejen ze společností, které na svých stránkách využívají algoritmy personalizující obsah, ale také ze samotných uživatelů. K prolomení filtrační bubliny, nebo alespoň snížení jejího vlivu na uživatele, existuje hned několik různých cest. Kromě prolomení vlastních zvyků - tedy aktivní snaze vyhledávat zdroje, které bychom za normál\-ních okolností nenavštívili, můžou uživatelé také pravidelně mazat své cookies nebo si vybírat platformy, které jsou ohledně fungování algoritmů na svých strán\-kách transparentnější než jiné.~\cite{Pariser2011}
    
    Je mimo jiné možné využít komplexních nástrojů, které se snaží uživatelům usnadnit cestu z filtračních bublin. Jedním takovým nástrojem je také aplikace Balance, kterou je možné nainstalovat si do prohlížeče Chrome. Toto rozšíření klasifikuje prohlížené stránky a zařazuje je podle toho, kde se pohybují v rámci politického spektra. Pokud se výsledky příliš kloní na jednu nebo druhou stranu, uživatel je o této skutečnosti informován a Balance mu dokonce navrhne zdroje, které mu mohou pomoci rozšířit jeho obzory. Tento nástroj je však primárně určený pro publikum Spojených států amerických, a proto je jeho uplatnění na české zdroje nefunkční. Pokud však jedinec bude čerpat i ze zahraničních médií, může mít nástroj jistou výpovědní hodnotu.~\citep{Munson13}
    
    Vedle filtračních bublin je v literatuře možné narazit také na pojmy jako jsou informační bubliny nebo echo chambers. Tyto jevy bývají často zaměňovány a proto je důležité pochopit, čím jsou definovány a jak se k sobě vzájemně vztahují. Z toho důvodu jsou jim věnovány následující podkapitoly.
%%%%% ===============================================================================

\section{Informační bubliny}
\label{sec:informacni-bubliny}
    Pojem filtrační bublina bývá často volně zaměňován s označením informační bublina. Děje se tak především v českém akademickém prostředí~\citep{Mudrovamastersthesis,Rivamastersthesis, Valentovamastersthesis}.
    
    Tento český překlad používá také~\cite{GregorVejvodova} ve své knize „Nejlepší kniha o fake news!!!“. Stejně jako~\citep{Mudrovamastersthesis,Rivamastersthesis,Valentovamastersthesis} používají v této knize Gregor a Vejvodová k definování pojmu informační bublina definici filtračních bublin podle~\cite{Pariser2012TheFB}. 
    
    Jedním z důsledků informačních bublin je podle nich polarizace názorů, ke které dochází, protože bublina znemožňuje existenci prostoru pro společnou debatu lidí s odlišnými postoji. Uživatelé jsou proto uvězněni v těchto bublinách, které jsou ze své podstaty ideálním prostředím pro šíření dezinformací. Jedinec je uzavřen v neustále se opakujícím informačním cyklu. Pokud začne věřit v pravdivost určité informace, začne vyhledávat další zdroje (podobného charakteru), které jeho teorii dále potvrdí. Na ně se pak i důsledkem personalizace nabalí další informační vrstvy potvrzující jedincův názor.~\citep{GregorVejvodova}
    
    Pojem informační bublina používají ve své práci také~\cite{Liao}, kteří jej v určitých pasážích používají namísto filtračních bublin. Termín informační bubliny popisují jako situaci, kdy jedinec vyhledává takové zdroje, které jsou konzistentní s jeho vlastními názory a postoji. Podle této jejich studie má na vznik informační bubliny vedle personalizace vliv také řada dalších faktorů. Jedním z nich může být například míra angažovanosti v daném tématu. 
    
    \setlength\parskip{5mm}
    
    \uv{\textit{Naše studie ukázala, že i když byly vedle sebe prezentovány protikladné pohledy, tak vyhledávání informací, které bylo doprovázené pocitem relevantní hroz\-by, vedlo k zřetelnější selektivní expozici názorově konzistentních informací. Tato zvýšená úroveň selektivní expozice vede z důvodu nižšího příjmu názorově podnět\-ných informací k nižší názorové změně. Nicméně vysoká míra angažovanosti v daném tématu může nad touto tendencí převážit do takové míry, že se lidé začnou vystavovat relativně vyváženému poměru názorově konzistentních a nekonzistentních informací. I přes tuto skutečnost má vysoká míra angažovanosti k danému tématu za následek spíše přiklonění se k názorově konzistentním informacím než k těm nekonzistentním, a z velké části zvyšuje vzdor vůči názorové změně.}}\footnote{Přeloženo z originálu: Our  study  showed  that,  even when  opposing  views  were presented  side-by-side,  information seeking  under perceived relevant threat led  to more pronounced selective exposure  to attitude consistent  information.  This  increased level of selective exposure also leads to less attitude change due  to  the  overall  less  reception  of  attitude  challenging information. However, high topic involvement can override this  tendency  such  that  people  seek  relatively  balanced exposure to attitude consistent and inconsistent information. Nonetheless,  high  involvement  with  the  topic  results  in more  preferential  evaluation  of  attitude  consistent information  over  attitude  inconsistent  one,  and  largely increases the resistance to attitude change.}
    
    \setlength\parskip{0mm}

%%%%% ===============================================================================
\section{Echo chambers}
\label{section:echo-chambers}
    Filtrační bubliny nejsou prvním pojmem, který by popisoval situaci, kdy je jedinec konfrontován pouze s názory konzistentními k jeho vlastnímu přesvědčení. Ještě před definováním pojmu filtrační bublina popsal na začátku 21. století~\cite{Sunstein07} podobný termín ve své knize Republika.com 2.0.. Jedná se o takzvané echo chambers\footnote{Česky překládáno například jako ozvěnové komory nebo komnaty ozvěn.}. Přestože v této konkrétní publikaci neuvádí jejich explicitní definici, echo chambers se věnuje do detailu.
    
    Tento termín popisuje prostředí, ve kterém jedinci slyší pouze názory, které se podobají jejich vlastním, tedy naslouchají ozvěně svých vlastních hlasů. Tato situace pak může vést ke skupinové polarizaci názorů. Spolu s echo chambers zmiňuje autor také information cocoons (informační kokony). Oba tyto pojmy nemusí být podle něj nutně spojovány pouze s internetem. Echo chambers a information cocoons jsou považovány za hrozbu demokracie a jsou známkou jejího špatného fungování.
    \cite{Sunstein17}
    
    Filtrační bubliny stejně jako echo chambers nemají pevnou definici, což komplikuje určení jejich vzájemného vztahu, rozdílných a společných charakteristik.~\cite{Bruns} 
    
    \cite{Bruns17} definuje echo chambers takto:
    
    \setlength\parskip{5mm}
    
    \uv{\textit{Echo chamber vzniká, když se skupina účastníků rozhodne záměrně vzájemně propojit na úkor lidí nacházejících se mimo tuto skupinu. Čím více je tato síť zformována (čím více spojení se utvoří uvnitř skupiny za vyloučení lidí zvenčí), tím více se izoluje od odlišných názorů vně skupiny, což umožňuje lepší cirkulaci již stávajících názorů uvnitř uskupení.}}\footnote{Přeloženo z originálu: An echo chamber comes into being where a group of participants choose to preferentially connect with each other, to the exclusion of outsiders. The more fully formed this network is (that is, the more connections are created within the group, and the more connections with outsiders are severed), the more isolated from the introduction of outside views is the group, while the views of its members are able to circulate widely within it.}
    
    Oba pojmy jsou tedy velmi úzce spjaty a často jsou vzájemně zaměňovány. Jejich hlavní rozdíl spočívá v behaviorálních znacích dané skupiny na jedné straně a strukturálních vlastnostech propojení účastníků na straně druhé.~\citep{Bruns17}
    
    \setlength\parskip{0mm}
    
    V echo chamber si účastníci dobrovolně volí vzájemný kontakt s ostatními ve skupině a čím více se mezi sebou propojují a jejich vztahy se utužují, tím více se uzavírají před lidmi mimo svoji skupinu a jejich názory mezi nimi cirkulují (Jako příklad může sloužit situace, kdy v prvním ročníku na nové škole se na začátku roku každý baví s každým, ale postupně se tvoří jednotlivé party, které čím dál více ztrácejí kontakt s ostatními spolužáky a interaguji pouze mezi sebou.).~\citep{Bruns17}
    
    Ve filtračních bublinách si lidé záměrně volí vzájemnou komunikaci s ostatními jedinci bez ohledu na systém a strukturu jejich vzájemného propojení. Čím více se vyhýbají komunikaci s dalšími účastníky, tím více dochází k utužování jejich vlastních názorů (např. v komentářích se rozvíjí diskuze na téma rostlinné stravy a její vyznavači si vyměňují argumenty pro bez ohledu na argumenty proti, které do debaty přináší zástupci karnivorů).~\citep{Bruns17}
    
    Echo chambers jsou apolitické, ale mohou ovlivňovat rozhodování zákonodárců - sbližovat strany, ale také fungovat proti konsensu~\cite{Jasny}.


%%%%% ===============================================================================


