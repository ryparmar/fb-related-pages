\chapter{Metodika}
\label{chapter:metodika}
    Součástí tohoto textu je obsahová analýza facebookových stránek, které se zabývají klimatickou krizí. Výzkumná část je postavena na třech výrocích či akcích Facebooku detailněji popsaných v kapitole~\ref{chapter:facebook} Sociální síť Facebook. Všechny tyto „výroky“ souvisí se vznikem filtračních bublin a jsou shrnuty do následujících bodů:
    \begin{enumerate}
        \item Facebooku usiluje o doporučování různorodého obsahu a tím předchází vzniku filtračních bublin. 
        \item Facebook se snaží omezovat šíření dezinformačních skupin.
        \item Stránky s větší mírou interakce se lépe šíří - vyšší míru interakce často mají stránky, které porušují zásady facebookové komunity. 
    \end{enumerate}
    
    Tato kapitola je krátkým úvodem do designu výzkumu, který vychází z těchto výroků. Kapitola stanovuje cíle a zasazuje výzkum do konkrét\-ního metodického rámce. 
 
 
\section{Cíl výzkumu}
\label{sec:cil-vyzkumu}
    Cílem výzkumu je zjistit, zda facebookový algoritmus napomáhá k prolamování informačních bublin na stránkách, které se zabývají klimatickou krizí.


\section{Výzkumná otázka}
\label{sec:vyzkumna-otazka}
    \setlength\parskip{5mm}
    
    První výzkumnou otázkou je: Pomáhá Facebook prolamovat filtračí bubliny doporučováním stránek s různým postojem ke klimatické krizi? 
    
    Druhou výzkumnou otázkou je: Jaká je frekvence doporučení stránek pro nebo proti klimatické krizi? 
    
    Třetí výzkumnou otázkou je: Jak se liší interakce na doporučovaných stránkách pro a proti klimatické krizi? 
    
    \setlength\parskip{0mm}


\section{Stanovení hypotéz}
\label{sec:stanoveni-hypotez}
    Jestliže se Facebook snaží předcházet vzniku filtračních bublin, mělo by se to projevit také v uskutečněném výzkumu. Abychom mohli dojít k jasným závěrům o filtarčních bublinách na Facebooku, byly stanoveny hypotézy, které vychází jak z cíle výzkumu, tak výzkumných otázek. 
    
    \setlength\parskip{5mm}
    
    H1: Facebook doporučuje všechny stránky o klimatické změně stejnou měrou bez ohledu na jejich postoj.
    
    \setlength\parskip{0mm}{}
    
    H2: Facebookový algoritmus posiluje vznik filtračních bublin na stránkách, které se věnují klimatické krizi.

\section{Motivace a význam}
\label{sec:motivace-vyznam}
    O filtračních bublinách se často hovoří v kontextu politické diskuze a rozhodování. Výskyt tohoto jevu však není omezen pouze na tuto konkrétní oblast, ale odráží se také na postojích ke klimatické krizi. 
    Ta se čím dál častěji stává tématem pro média, neboť se její dopady stále ve větší míře projevují v nejrůznějších koutech planety - příkladem mohou být rozsáhlé požáry v Austrálii na začátku roku 2020.~\cite{tarabay10}
    
    Rostoucí skepsi veřejnosti, která panuje vůči klimatické krizi, napomáhají filtrační bubliny, které utvrzují jedince v jeho vlastním postoji, a znemožňují mu tak získat různorodé podněty, které by mu umožnily informovanou změnu názoru.~\cite{carmichael} Společnost se tak rozděluje na dva tábory, které jen stěží docházejí ke konsensu.~\cite{WILLIAMS2015126} Ten je pro klimatickou krizi důležitý zejména proto, že skepse vůči tomuto tématu může mít negativní efekt na schvalování a důvěru vůči environmentálním opatřením, které se usilují o zmírnění negativních dopadů klimatické změny.~\citep{AKLIN2014173}
    
    \section{Výzkumná metoda}
    Použitou metodou bude obsahová analýza facebookových stránek s negativním a pozitivním postojem ke klimatické krizi. Obsahová analýza je jednou z nejdůležitějších výzkumných technik v humanitních vědách, která pohlíží na data jako na určitý druh komunikace s významem a kontextem, který je specifický pro dané publikum příjemců.~\cite{krippendorff2018content}
    
    \cite{Neuendorf} ve své knize uvádí, že obsahová analýza je numerickým procesem, jehož cílem je shrnutí dané skupiny zpráv. Nejedná se o abstraktní, ani detailní popis zprávy či skupiny zpráv. Jedná se tedy o kvantitativní analýzu. Přesněji o definuje obsahovou analýzu takto: 
    \setlength\parskip{5mm}
    
    \textit{
    „Obsahová analýza je shrnující, kvantitativní analýza zpráv, která se opírá o vědeckou metodu (včetně pozornosti vůči objektivitě-intersubjektivitě, předchozímu designu, reliability, validity, zobecnitelnosti, replikovatelnosti a testování hypotéz) a není omezena na typy proměnných, které lze měřit, nebo na kontext, ve kterém jsou zprávy vytvářeny či prezentovány.“}\footnote{Přeloženo z originálu: Content analysis is a summarizing, quantitative analysis of messagges that relies on the scientific method (including attention to objectivity-intersubjectivity, a prior design, reliability, validity, generalizability, replicability, and hypothesis testing) and is not limited as to the types of variables that may be measured or the context in which the messages are created or presented.} (s. 10)
    
    \setlength\parskip{0mm}

