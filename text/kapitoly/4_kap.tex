\chapter{Klimatická změna}
\label{chapter:klimaticka-zmena}
%\textcolor{green}{Kapitola nekompletní - 
    %Exituje nějaké skupina popíračů? Je to fenomén? Jak velký problém se z toho stal? je to vůbec problém? 
    %Filtrační bubliny a klimatická krize. Jaké jsou další studie? }

    Jak už bylo řečeno, kolem klimatické změny koluje spoustu dezinformací.~\citep{kolmes2011climate} Proto je nezbytné vymezit, co tento pojem znamená, co ho způsobuje a naopak s jakými pojmy bývá chybně zaměňován. Teoretické vymezení klimatické změny je také základem pro definici facebookových stránek, které se věnují klimatické krizi v následujících kapitolách. 
    
    Klimatická změna je definována dlouhodobou změnou v průměrných vzorcích počasí, které definují lokální, regionální a globální klima země. Pojem je často volně zaměňován s globálním oteplováním, nicméně hlavním rozdílem je, že klimatická změna označuje jak přirozené změny klimatu, tak ty, které jsou způsobeny člověkem.
    
    Klimatické změny, které byly vypozorovány od začátku 20. století, jsou převáž\-ně přisuzovány lidskému vlivu. Jedná se o spalování fosilních paliv, které způsobuje uvolňování skleníkových plynů do atmosféry a následně její rychlejší oteplování. Od před-idustriálního období se planeta oteplila o celý 1 °C, a toto číslo se v současnosti zvyšuje přibližně o 0.2 stupně každé desetiletí. 
    
    Je také důležité nezaměňovat klima s počasím. Počasí totiž označuje pouze atmosférické podmínky, které jsou lokální a krátkodobé povahy - vyznačují se sněhem, deštěm, mraky, větrem atd. Naproti tomu klima je dlouhodobého charakteru a pracuje s regionálními nebo celosvětovými hodnotami teploty, vlhkosti, srážkového úhrnu za dlouhé časové období - měsíce, roky, dekády.~\citep{nasa_2021}
    
    Jedním z důsledků klimatické změny je méně předvídatelné počasí, což může být obzvláště problematické pro státy, které jsou závislé na zemědělství a nekontrolovatelné změny počasí jim ovlivňují úrodu. Klimatické změny jsou také spojovány s dalšími živelnými událostmi jako jsou hurikány, záplavy, průtrže mračen nebo sněhové bouře. Na severu pak kvůli klimatickým změnám dohází k rychlejšímu tání ledovců, které způsobují stoupání hladiny moří v různých částech světa. 
    
    Jak už bylo zmíněno, klimatická změna je dvojí povahy - přirozená a způsobená lidmi. Ke změně klimatu tedy kontinuálně docházelo i dříve v historii. Jednalo se o pomalé a postupné změny, které trvaly stovky i tisíce let. Klimatické změny, které pozorujeme v současnosti, se však objevují v mnohem rychlejším tempu.~\citep{nationalgeographicsociety_2019}
    
    A právě kvůli rychlým změnám klimatu, které jsou (mimo jiné) charakterizovány nepředvídatelnými výkyvy počasí, zvyšováním hladiny oceánů nebo rizikem nedostatečných vodních zásob, mluvíme o klimatické krizi.~\citep{vitek_2020}
    
    Co je klimatická změna se nyní může zdát jako poměrně jasné, přesto existence antropocentrické (lidmi způsobené) klimatické změny vyvolává mezi veřejností živou debatu. Mimo jiné o tom, čím je charakterizována tato diskuze, co ji způsobuje a jak se vyznačují obě zainteresované strany pojednává následující kapitola.
    
%%%%% ===============================================================================
\section{Klimatická krize - filtrační bubliny a polarizace}
\label{sec:klimaticka-krize-bubliny}
    Přestože současná klimatická krize není novinkou a o její existenci se diskutuje již přibližně třicet let, za tuto dobu se nejen nepodařilo přijít na účinná řešení, která by klimatickou změnu zcela eliminovala (nebo alespoň výrazně zpomalila), ale dokonce se ani nepodařilo v tomto tématu dosáhnout konsensu u široké veřejnosti - jejíž jednotná kooperace je pro řešení problému důležitá. Ideologické odmítání klimatické krize se tak stalo předmětem řady studií napříč obory v posledních dvaceti letech.~\citep{almiron2019rethinking}
    
    Každý z nich nachází pro popírání klimatické krize své vlastní vysvětlení, ve kterém mají hlavní roli různí činitelé - od ziskuchtivého horního 1 \%, které ovlivňuje trh, až po jednotlivce, kteří nejsou schopni pochopit hloubku problému, a vyrovnat se s ním v kontrastu k jejich vlastním problémům každodenní reality.~\citep{mathers2020anthropology}
    
    Klimatická krize není izolovaným tématem, ale je zasazena do širšího kontextu toho, jaký názor má jedinec na jednotlivé politické otázky. To ovlivňuje způsob, jakým bude na celou problematiku nahlížet. 
    
    \setlength\parskip{5mm}
    
    \textit{„Studie ukazuje, že světonázor má malý, ale nezanedbatelný vliv na to, jakým způsobem je narativ klimatické krize zapamatován a převyprávěn. Výzkum mentál-ní reprezentace příběhu klimatické krize skrze třídění úkolů/klastrové analýzy odhalil, že ačkoliv obecná struktura zůstává stejná napříč světonázory, pouto mezi komponentem problému (krize kvůli klimatické změně) a navrhovaným řešením problému (strategie v boji s klimatickou změnou) se systematicky liší v závislosti na světonázor publika nebo mluvčího.“}\footnote{Přeloženo z: The present study shows that world views exert a small though non-negligible influence on how climate
    change narratives are remembered and retold. An examination of the mental representation of a climate change story via a
    sorting task/cluster analysis approach revealed that although the general story structure is very similar across world views, the
    link between the problem component (a crisis due to climate change) and the proposed problem solutions (strategies to
    counteract climate change) varies systematically as a function of the audience’s world view and of the speaker’s world view.}~\citep{bohm2019remembering}
    
    Klimatická krize proto může způsobovat vysokou míru polarizace. Lidé se dělí na „aktivisty“ nebo „skeptiky/popírače“. Diskuze na toto téma je charakterizována silnou homofilií založenou na daném postoji a rozdělením do stejně smýšlejících komunit, ze kterých uživatelé příliš nevystupují, aby se zapojili do opoziční diskuze. Lidé se silnými názory jsou také v diskuzích nejvíce slyšet.~\citep{WILLIAMS2015126} Proto je téma globální klimatické krize ideální živnou půdou pro vznik filtračních bublin a echo chambers.
    
    \setlength\parskip{0mm}
    
    Přestože „aktivisti“ i „skeptici/popírači“ mají jako svůj primární zdroj informací televizní zprávy, jejich sekundární zdroj se liší. Pro 17.8 \% „aktivistů“ jsou to televizní zprávy, pro 18.2 \% „skeptiků/popíračů“ jsou to exlusivně online dostupná média (jako např. Buzfeed, Huffington Post atd.). Klimatičtí skeptici také dávají větší přednost informacím z facebookové zdi nebo informacím, které sdílejí jejich přátelé a stránky tradičních novinových zdrojů (washingtonpost.com, nytimes.com atd.) jsou pro ně až na posledním místě. Dokonce jsou i za skupinou „další“ (blíže nespecifikované alternativní zdroje informací), které jsou sekundárním zdrojem až pro 10 \% z nich.  
    
    V otázkách klimatické změny pak můžou hrát právě zdroje pro čerpání informací důležitou roli.~\cite{carmichael} se ve své studii zabývají zvyšující se předpojatostí veřejnosti vůči problémům tohoto celospolečenského problému. Jejich zjištění potvrzuje hypotézu, že média mohou jedince utvrdit v jeho existujícím názoru, pokud přináší informace, které jsou s jeho vlastním pohledem konzistentní. Účastníci facebookové diskuze na téma klimatické krize mají tendenci soustředit se na jim blízký narativ a přehlížet jakýkoli jiný. Tím se vytváří struktura podobná echo chambers.
    
    Čím větší jsou rozdíly mezi sentimenty jednotlivých účastníků diskuze, tím větší je také polarizace~\citep{Zollo2019}. Stejné výsledky ukázala také studie realizovaná na sociální síti Twitter, která sledovala debatu okolo vydání IPPC\footnote{Integrated Pollution Prevention and Control v češtině Integrovaná prevence a omezování znečištění. Zpráva se zabývá právě klimatickou krizí.} zprávy z roku 2013. Tato studie potvrzuje, že se uživatelé častěji zapojují do debaty s lidmi, pokud sdílejí společný názor~\citep{pearce2014climate}.
    
    Podle dalšího průzkumu na vzorku americké veřejnosti se klimatičtí popírači vyznačují nižší mírou důvěry vůči vědecké komunitě oproti lidem, kteří v krizi věří.~\citep{krishna2021understanding} Neúměrné množství času a úsilí je tak v diskuzích na sociálních sítích věnováno nesouhlasu, pochybám nebo snaze dojít ke konsensu nad vědeckými otázkami klimatické krize, spíše než pokusu o nalezení přijatelných opatření proti tomuto problému~\citep{martin2014rebalancing}.
    
    Názory na klimatickou krizi, které jsou v rozporu s vědeckými výzkumy, se častěji objevují v zemích, kde odborníci na toto téma konsensu dosáhli. Hlasy klimatických skeptiků mohou být ve veřejné debatě přehlíženy, a proto tito lidé vyjadřují své názory v komentářích na internetu, kde se navzájem ve svých názorech utvrzují.~\citep{walter2018echo} 
    
    Facebook se v rocoe 2021 vyjádřil, že se pokusí bojovat proti dezinformacím, které jsou spojeny s klimatickou krizí a spustil na platformě ve Spojeném králov-ství testování nového systému, který by měl pomáhat vyvracet dezinformace. U některých příspěvků by měly být připnuty štítky, které budou uživatele odkazovat do repozitáře ověřených informací o klimatické změně. Podobný systém nedávno použili pro americké prezidentské volby 2020. Na tomto opatření by se měli podílet odborníci z celého světa. Společnost momentálně bojuje proti dezinformacím snižováním dosahu nepravdivých příspěvků.~\citep{hern_2021}
    
    Že je edukace v oblasti klimatické krize a snaha přesvědčit popírače o akutnosti tohoto tématu komplexním problémem, který nemá jednoduché řešení, dokazuje to, jak dlouho se už akademická obec tématikou zabývá bez většího výsledku. Například~\cite{nisbet2009communicating} už v roce 2009 navrhoval deduktivní soubor mentálních „zásuvek“ a interpretačních příběhů, který měl pomoci spojit publikum v otáz\-kách klimatické krize, a utvářet tak chování jednotlivců nebo mobilizovat kolektivní činy. Přesto i v roce 2021, o dvanáct let později, se ukazuje, že téma je stále aktuální. 
    
    Také proto se tato problematika stala námětem pro výzkum, který byl popsán v následujících kapitolách této práce. Ani v současné době, kdy máme v živé paměti rozsáhlé požáry v Australii a svět se potýká s globální pandemickou krizí, se nedaří veřejnost v otázkách klimatické krize sjednotit a dosáhnout konsensu~\citep{tarabay10,leyen_ghebreyesus_2020}. Následující kapitoly snad pomohou vyjasnit některé otazníky, které mohou viset nad fungováním filtračních bublin na facebookových stránkách, jež se zabývají klimatickou krizí.

%%%%% ===============================================================================


